\documentclass[11pt]{article}
\usepackage{amssymb}
\usepackage{amsthm}
\usepackage{enumitem}
\usepackage{amsmath, physics}
\usepackage{bm}
\usepackage{adjustbox}
\usepackage{mathrsfs}
\usepackage{graphicx}
\usepackage{siunitx}
\usepackage[mathscr]{euscript}
\usepackage{tikz}
\usepackage{float}

\usetikzlibrary{automata, positioning, arrows}
\tikzset{
->, % makes the edges directed
>=stealth', % makes the arrow heads bold
node distance=3cm, % specifies the minimum distance between two nodes. Change if necessary.
every state/.style={thick, fill=gray!10}, % sets the properties for each ’state’ node
initial text=$ $, % sets the text that appears on the start arrow
}

\title{\textbf{Solved selected problems of Introduction to the Theory of Computation by Michael Sipser.}}
\author{Franco Zacco}
\date{}

\addtolength{\topmargin}{-3cm}
\addtolength{\textheight}{3cm}

\newcommand{\N}{\mathbb{N}}
\newcommand{\Z}{\mathbb{Z}}
\newcommand{\Q}{\mathbb{Q}}
\newcommand{\R}{\mathbb{R}}
\newcommand{\diam}{\text{diam}}
\newcommand{\cl}{\text{cl}}
\newcommand{\bdry}{\text{bdry}}
\newcommand{\inter}{\text{int}}
\newcommand{\hatx}{\bm{\hat{x}}}
\newcommand{\haty}{\bm{\hat{y}}}
\newcommand{\hatz}{\bm{\hat{z}}}
\newcommand{\hatrho}{\bm{\hat{\rho}}}
\newcommand{\hatphi}{\bm{\hat{\phi}}}
\newcommand{\hatr}{\bm{\hat{r}}}
\newcommand{\hattheta}{\bm{\hat{\theta}}}
\newcommand{\esp}{\text{\textvisiblespace}\hspace{0.1em}}

\theoremstyle{definition}

\begin{document}
\maketitle
\thispagestyle{empty}

\section*{3 The Church-Turing Thesis}
\subsection*{3.1 Turing Machines}

\begin{proof}{\textbf{3.1}}
\begin{itemize}
    \item [\textbf{a.}]
    \begin{align*}
        &q_1 0 \esp\\
        &\esp q_2 \esp\\
        &\esp \esp q_{\text{accept}}
    \end{align*}
    \item [\textbf{b.}]
    \[
    \begin{aligned}
        &q_1 0 0 \esp     &  &\esp q_2 x \esp\\
        &\esp q_2 0 \esp  &  &\esp x q_2\esp\\
        &\esp x q_3 \esp  &  &\esp x \esp q_{\text{accept}}\\
        &\esp q_5 x \esp  & \\
        &q_5 \esp x \esp  & \\
    \end{aligned}
    \]
    \item [\textbf{c.}]
    \[
    \begin{aligned}
        &q_1 0 0 0\esp      & \\
        &\esp q_2 0 0\esp   & \\
        &\esp x q_3 0 \esp  & \\
        &\esp x 0 q_4 \esp  & \\
        &\esp x 0 \esp q_{\text{reject}}
    \end{aligned}
    \]
    \item [\textbf{d.}] 
    \[
    \begin{aligned}
        &q_1 0 0 0 0 0 0\esp    & &\esp x 0 x 0 q_5 x \esp & &\esp x q_2 0 x 0 x \esp \\
        &\esp q_2 0 0 0 0 0\esp & &\esp x 0 x q_5 0 x \esp & &\esp x x q_3 x 0 x \esp \\
        &\esp x q_3 0 0 0 0\esp & &\esp x 0 q_5 x 0 x \esp & &\esp x x x q_3 0 x \esp \\
        &\esp x 0 q_4 0 0 0\esp & &\esp x q_5 0 x 0 x \esp & &\esp x x x 0 q_4 x \esp \\
        &\esp x 0 x q_3 0 0\esp & &\esp q_5 x 0 x 0 x \esp & &\esp x x x 0 x q_4 \esp \\
        &\esp x 0 x 0 q_4 0\esp & &q_5 \esp x 0 x 0 x \esp & &\esp x x x 0 x \esp q_{\text{reject}} \\
        &\esp x 0 x 0 x q_3\esp & &\esp q_2 x 0 x 0 x \esp \\
    \end{aligned}
    \]
\end{itemize}
\end{proof}
\begin{proof}{\textbf{3.2}}
    \begin{itemize}
        \item [\textbf{a.}]
        \begin{align*}
            &q_1 11 \esp\\
            &x q_3 1 \esp\\
            &x 1 q_3 \esp\\
            &x 1 \esp q_{\text{reject}}
        \end{align*}
        \item [\textbf{b.}]
        \[
        \begin{aligned}
            & q_1 1 \# 1 \esp     &  & x q_1 \# x \esp\\
            & x q_3 \# 1 \esp     &  & x \# q_8 x \esp\\
            & x \# q_5 1 \esp     &  & x \# x q_8 \esp\\
            & x q_6 \# x \esp     &  & x \# x \esp q_{\text{accept}}\\
            & q_7 x \# x \esp
        \end{aligned}
        \]
        \item [\textbf{c.}]
        \[
        \begin{aligned}
            & q_1 1 \# \# 1 \esp\\
            & x q_3 \# \# 1 \esp\\
            & x \# q_5 \# \esp\\
            & x \# \# q_{\text{reject}}\esp
        \end{aligned}
        \]
        \item [\textbf{d.}] 
        \[
        \begin{aligned}
            & q_1 1 0 \# 1 1 \esp  &  & x q_1 0 \# x 1 \esp\\
            & x q_3 0 \# 1 1 \esp  &  & x x q_2 \# x 1 \esp\\
            & x 0 q_3 \# 1 1 \esp  &  & x x \# q_4 x 1 \esp\\
            & x 0 \# q_5 1 1 \esp  &  & x x \# x q_4 1 \esp\\
            & x 0 q_6 \# x 1 \esp  &  & x x \# x 1 q_{\text{reject}}\esp\\
            & x q_7 0 \# x 1 \esp  &  & \\
            & q_7 x 0 \# x 1 \esp  &  & 
        \end{aligned}
        \]
        \item [\textbf{e.}] 
        \[
        \begin{aligned}
            & q_1 1 0 \# 1 0 \esp  &  & x q_1 0 \# x 0 \esp  &  & x x q_1 \# x x \esp\\
            & x q_3 0 \# 1 0 \esp  &  & x x q_2 \# x 0 \esp  &  & x x \# q_8 x x \esp\\
            & x 0 q_3 \# 1 0 \esp  &  & x x \# q_4 x 0 \esp  &  & x x \# x q_8 x \esp\\
            & x 0 \# q_5 1 0 \esp  &  & x x \# x q_4 0 \esp  &  & x x \# x x q_8 \esp\\
            & x 0 q_6 \# x 0 \esp  &  & x x \# q_6 x x \esp  &  & x x \# x x \esp q_{\text{accept}}\\
            & x q_7 0 \# x 0 \esp  &  & x x q_6 \# x x \esp\\
            & q_7 x 0 \# x 0 \esp  &  & x q_7 x \# x x \esp
        \end{aligned}
        \]
    \end{itemize}
\end{proof}

\cleardoublepage
\begin{proof}{\textbf{3.3}}
\begin{itemize}
    \item [($\Rightarrow$)] If a language $L$ is decidable then there is a
    deterministic Turing machine which decides it, but this machine is
    nondeterministic as well so there is a nondeterministic Turing machine
    which decides $L$.

    \item [($\Leftarrow$)]
    We will use the proof to Theorem 3.16 to prove that every nondeterministic
    Turing machine that decides a language has an equivalent deterministic
    Turing machine that decides the language.

    Suppose there is some nondeterministic Turing machine $N$ that decides a
    language $L$.
    
    We need a multitape deterministic Turing machine $D$ where
    as we did in Theorem 3.16, the first tape (input tape) contains the input
    string, the second tape (simulation tape) contains a copy of $N$'s tape on
    some branch of its nondeterministic computation, and tape 3 (address tape)
    keeps track of $D$'s location in $N$'s nondeterministic finite computation
    tree.

    Then $D$ is described as follows
    \begin{itemize}
    \item [\textbf{1.}] Initially tape 1 contains the input $w$, and tapes 2
    and 3 are empty.
    \item [\textbf{2.}] Copy tape 1 to tape 2.
    \item [\textbf{3.}] Use tape 2 to simulate $N$ with input $w$ on one branch
    of its nondeterministic computation.
    Before each step of $N$ consult the next symbol on tape 3 to determine
    which choice to make among those allowed by $N$'s transition function.
    If no more symbols remain on tape 3 or if this nondeterministic choice is
    invalid, abort this branch by going to stage 4.
    Also, go to stage 4 if a rejecting configuration is encountered.
    If an accepting configuration is encountered, accept the input.
    \item [\textbf{4.}] Replace the string on tape 3 with the lexicographically
    next string.
    Simulate the next branch of $N$'s computation by going to stage 2.
    \item [\textbf{5.}] If we tried every possible string on tape 3 and we
    didn't reach any accept configuration, then reject the input.
    \end{itemize}
    Therefore, since $D$ cannot loop forever because $N$'s tree is finite and
    hence we must reach an accept or reject state then we get a deterministic
    Turing machine that decides the language $L$ i.e. the language is
    decidable.
\end{itemize}
\end{proof}

\cleardoublepage
\begin{proof}{\textbf{3.4}}\\
An \textbf{enumerator} is a 7-tuple,
$(Q, \Sigma ,\Gamma,\delta, q_0,q_{accept}, q_{reject})$, where
$Q, \Sigma, \Gamma$ are all finite sets and
\begin{itemize}
\item [\textbf{1.}] $Q$ is the set of states,
\item [\textbf{2.}] $\Sigma$ is the input alphabet,
\item [\textbf{3.}] $\Gamma$ is the printer alphabet,
\item [\textbf{4.}] $\delta: Q \times \Sigma \to Q \times \Sigma
\times \mathcal{P}(\Gamma)$ is the transition function,
\item [\textbf{5.}] $q_0 \in Q$ is the start state,
\item [\textbf{6.}] $q_{accept} \in Q$ is the accept state, and
\item [\textbf{7.}] $q_{reject} \in Q$ id the reject state, where
$q_{reject} \neq q_{accept}$.
\end{itemize}
An enumerated language $L$ is the language that an enumerator $E$ prints out
hence $L \subset \mathcal{P}(\Gamma)$.
\end{proof}

\cleardoublepage
\begin{proof}{\textbf{3.5}}
\begin{itemize}
\item [\textbf{a.}] Given that the transition function is defined as
$\delta: Q \times \Gamma \to Q \times \Gamma \times \{L,R\}$ and 
$\esp \in \Gamma$ then a Turing Machine can write $\esp$ on its tape.

\item [\textbf{b.}] No, the tape alphabet $\Gamma$ cannot be the same as the
input alphabet $\Sigma$ since by definition $\esp \in \Gamma$ but
$\esp \not\in \Sigma$.

\item [\textbf{c.}] Yes, this could happen when the head is on the leftmost
square of the tape, then if the transtion function asks the head to go to the
left the head will stay where it is hence the head will be in the same location
in two successive steps.

\item [\textbf{d.}] No, a turing machine by definition contains at least
$q_{reject}$ and $q_{start}$ where $q_{reject} \neq q_{accept}$.
\end{itemize}
\end{proof}
\begin{proof}{\textbf{3.7}}
The machine $M_{bad}$ is not a legitimate Turing machine for two reasons.

First, given that the machine tries all possible settings of $x_1, ..., x_k$
to integer values, we must store these integer values somehow, but a Turing
machine can only have finite alphabets $\Gamma$ and $\Sigma$.

Finally, this machine may not halt, because the polynomial may not be solvable,
and hence we may loop forever.
\end{proof}

\cleardoublepage
\begin{proof}{\textbf{3.8}}
\begin{itemize}
    \item [\textbf{a.}]
    $M_1 = \{w~|~w \text{ contains an equal number of 0s and 1s}\}$\\
    On input string $w$:
    \begin{itemize}
        \item [\textbf{1.}] Scan the first element of the tape from left to
        right if it's a 0 then cross it off and continue scanning the tape
        until you find a 1 and cross it off. If the first element is a 1 then
        cross it off and continue scanning the tape until you find a 0 and
        cross it off.
        \item [\textbf{2.}] If you reach the end of the tape (a symbol $\esp$)
        and you do not find either a 1 or a 0 depending on the case you are
        in, then reject.
        \item [\textbf{3.}] Return to the second element and repeat the process
        described in \textbf{1.} and \textbf{2.}
        Do this for all slots in the tape.
        If you encounter a crossed-off element on the tape continue to the
        next slot.
        \item [\textbf{4.}] If there are no more slots with 0s or 1s then
        accept, otherwise reject.
    \end{itemize}
    \item [\textbf{b.}]
    $M_2 = \{w~|~w \text{ contains twice as many 0s as 1s}\}$\\
    On input string $w$:
    \begin{itemize}
        \item [\textbf{1.}] Scan the tape from left to right until you find a 1.
        If there are no 1s but there are 0s, reject. Cross off the 1 you found.
        \item [\textbf{2.}] Return to the head of the tape and scan from left
        to right until you find a 0 and cross it off, continue scanning until
        you find another 0 and cross it off. If while trying to execute this
        step there are no more 0s, reject.
        \item [\textbf{3.}] Continue the process described in \textbf{1.} and
        \textbf{2.} for every 1 you find. If there are no more 1s but there are
        still 0s, reject. If while executing this step there are no more 0s and
        1s. accept.
    \end{itemize}
    \item [\textbf{c.}]
    $M_2 = \{w~|~w \text{ does not contain twice as many 0s as 1s}\}$\\
    On input string $w$:
    \begin{itemize}
        \item [\textbf{1.}] Scan the tape from left to right until you find a 1.
        If there are no 1s but there are 0s, accept. Cross off the 1 you found.
        \item [\textbf{2.}] Return to the head of the tape and scan from left
        to right until you find a 0 and cross it off, continue scanning until
        you find another 0 and cross it off. If while trying to execute this
        step there are no more 0s you can cross off, accept.
        \item [\textbf{3.}] Continue the process described in \textbf{1.} and
        \textbf{2.} for every 1 you find. If there are no more 1s but there are
        still 0s, accept. If while executing this step there are no more 0s and
        1s. reject.
    \end{itemize}
\end{itemize}
\end{proof}

\end{document}
